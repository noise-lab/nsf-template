\section{Broader Impacts}\label{sec:impacts}

This project aims to education the next generation of computer scientists on
topics concerning the applications of machine learning to cybersecurity.  As
the focus on the project is education, many of the project's outcomes
naturally already touch on broader impacts, through student education. Below,
we expand on various themes and discuss how the research and education efforts
that we described in previous sections can be translated to further broader
impacts.


\paragraph{Student and researcher mentorship.} To improve education in ways
that broaden participation in computing, the PIs will create interconnected
programs that merge sustained engagement with diverse students in Chicago
Public Schools, provide summer research experiences, and carefully scaffold
these research experiences with on-ramp immersions that lower barriers to
participation.  

To provide meaningful research experiences to students, the PIs will annually
mentor a diverse group of students, ranging in level from high school to
post-graduate as a way to encourage future participation in computing
programs. These efforts will benefit from engaged partnerships with
organizations to aid in recruitment and tracking student outcomes, including
established relationships between the University of Chicago computer science
department and other organizational units, including: (1)~The UChicago
Collegiate Scholars Program; (2) the Office of Special Programs (OSP) College
Prep; (3)~The Fisk-Vanderbilt Master's-to-PhD Bridge Program for
underrepresented students.

PIs have a successful track record in mentoring women and underrepresented
minorities. For example, PI Feamster currently advises three female Ph.D.
students, two of whom are underrepresented minorities; PI Schmitt currently
advises two female Masters students, both of whom are underrepresented minorities. In the past, students mentored by
the PIs have won various awards. For example, both PIs mentored Annie
Edmundson a Ph.D. student at Princeton who won the ACM Student Research
Competition at ACM SIGCOMM. The PIs are involved with various efforts around
the department to increase inclusion and diversity.  Both PIs regularly
involve undergraduate students in research, both during the academic year and
summers.  The research performed by undergraduates has often resulted in
peer-reviewed publications (including as first author on top-tier venues such
as SIGCOMM and IMC).  Many of the PIs' undergraduates have continued either to
top graduate programs or highly desirable companies. Another aspect of
education is providing students with internship opportunities. The PI will
seek out real-world opportunities at government offices, companies, and civil
society organizations to train students.  When hiring, the PI will follow best
practices for encouraging diversity in hiring, including interviewing a
diverse set of candidates and advertising in forums that serve
under-represented minorities (\eg, Tapia and Grace Hopper).

\paragraph{Outreach to under-represented high school students.} To further
engage under-represented minorities in the project, the PIs will integrate
results from this project into K-12 education programs, such as the University
of Chicago's Upward Bound program, where PI Feamster currently teaches an
``Applications of Machine Learning to Networking'' summer course to
under-represented students across the Chicago Public School system.  This
program  has allowed the PI to directly engage with high school students
enrolled in Chicago Public Schools (77\% economically disadvantaged and 83\%
underrepresented minority) and their parents for an annual series of 10
seminars and 6 weekend workshops hosted at UChicago on our project's key
themes.  This on-ramp will introduce core computing skills, build an
understanding of the research process (e.g., how to ask a good research
question), and help the students set productive routines in unstructured
environments.  Perhaps more importantly, it will help build a community and
support network among each year's student cohort and the PIs through near-peer
mentoring and informal interactions.  Following this immersion, the students
will work for the rest of the summer with their respective co-PIs.

\paragraph{Course development.} The PI will develop new course materials
pertaining to Internet security, both for courses at the University of Chicago
and in the form of publicly accessible course material, as in Massive Open
Online Courses (MOOCs), which PI Feamster has developed in the past. The PI
already teaches several courses where topics from the research may be
integrated, including undergraduate Information Security, Networking, Many
University of Chicago undergraduate students also perform independent
research. The PI will generate project ideas for independent work projects and
undergraduate theses and will invite students to undertake these projects. A
major expected outcome from this project is to create course and curriculum
materials that can be used by large numbers of students, both at the
University of Chicago and at other institutions. As described in earlier
sections of the proposal, the PIs plan to develop these materials into content
that can be used in textbooks, online courses, professional development
materials, and other outlets. 
