\section{Results from Prior NSF Support}

\paragraph{Feamster:}
\textbf{CPS: Medium: Detecting and Controlling Unwanted
  Data Flows in the Internet of Things} \textit{1739809} (PI: Feamster);
10/1/18--9/30/22, \$875,000.
This
project develops technologies that ensure that IoT smart devices remain secure
and protect user privacy in the face of the widespread deployment of connected
smart devices.
\textbf{Intellectual Merit:} This project advances the theory and practice of network traffic
analysis and anomaly detection to secure IoT deployments. The project
has resulted in publications at top-tier
conferences~\cite{hooman2019:ccs,chu2018:iot:iotj,zheng2018:iot:cscw,feamster2018:iot:ctlj,doshi2018:iot,apthorpe2018:iot:imwut,datta2018:iot,acar2018:iot,weinberg2019:iot:www}, including studies of how home network traffic can
expose users to new privacy risks, new methods for modeling user attitudes and
norms about
privacy in smart homes, with applications to using contextual integrity,
and a broader understanding of how smart home
devices from appliances to smart
TVs may
collect private user data. They also include broader research on IoT
security and
privacy~\cite{chu2018:iot:iotj,zheng2018:iot:cscw,feamster2018:iot:ctlj,doshi2018:iot,apthorpe2018:iot:imwut,datta2018:iot,acar2018:iot,weinberg2019:iot:www}.
\textbf{Broader Impacts:} The research
has produced open-source software (IoT Inspector) to help smart home users better understand
how devices collect and share private data and the largest
labeled dataset of smart home device traffic with data from about 10,000
homes.
