\begin{center} 
    {\large \bf \TITLE } \\ 
    {\bf Postdoc Mentoring Plan}
\end{center} 

Postdoc mentoring will include the
following activities:

\begin{enumerate}
    \itemsep=-1pt
    \item {\bf Orientation.} PIs will offer each postdoctoral researcher an
        initial orientation discussing mutual expectations including: the
        amount of independence the Postdoctoral Researcher requires,
        interaction with coworkers, productivity including the importance of
        scientific publications, work habits, and documentation of research
        methodologies and experimental details.

    \item {\bf Career counseling.} PIs will offer the postdoctoral researchers
        career counseling providing the postdoctoral researcher with the
        skills, knowledge, and experience needed to excel in his/her chosen
        career path.  We will also encourage the postdoctoral researcher to
        discuss career options with other faculty at our respective
        institutions. 

\item {\bf Experience with funding applications.} When possible and
    appropriate, the PIs will involve postdoctoral researchers in the
        preparation of grant proposals, providing the opportunity to learn
        best practices in proposal preparation including: identification of
        key research questions, definition of objectives, description of
        approach and rationale, and construction of a work plan, timeline, and
        budget.

    \item {\bf Writing and presentation skills. }
       We will encourage the postdoc to write and publish scientific articles,
        as well as other reports and publications.  We will ask the
        postdoctoral researchers to present and lead discussions on scientific
        research at project meetings and present the results of their research
        to related research groups on campus. The PIs will also coach the
        postdoctoral researchers for external presentations at conferences, to
        program managers, and for other face-to-face meetings. 

\item {\bf Mentoring experience.} 
    Joint mentoring of graduate student interns work on the proposed project
    to expose the postdoctoral researcher to effective techniques in working
        with and managing students.

    \item {\bf Scientific guidance.} The fast pace of the proposed scientific
        program requires effective communication and a tight coordination of
        the effort.  The progress of the postdoc will be carefully followed by
        the PI, usually on a daily basis.  Group meetings take place at least
        weekly to discuss progress and plans. Additionally, the PIs will offer
        the postdoctoral researchers regular instruction in professional
        practices, including: fundamentals of the scientific method, proper
        etiquette in dealing with sensitive data and other standards of
        professional practice.

    \item {\bf Job search assistance.} The PIs will provide postdoctoral
        researchers with assistance with career application materials, including (as
        applicable) preparation of academic curriculum vitae, faculty
        applciation materials, and related support required to enable a
        successful academic job search.

    \item {\bf Technology transfer and industry connections.} Technology
        transfer activities will include regular contact with partner
        companies involved in the project.  As applicable, the postdoctoral
        researcher will be given an opportunity to become familiar with the
        research-industry relationship, including confidentiality requirements
        where necessary.

    \item {\bf Biannual assessment.} To assess the success of the above plan,
        the PI will meet with the postdoc twice a year for an additional,
        formal mentoring session. Weaknesses and strengths, career goals, research
        progress, work/life balance, interaction with students and collaborators, as
        well as other issues that may arise will be discussed, and help will be
        offered.

    \item {\bf Networking and career development.} The postdoc will be given
        opportunities to network with other scholars at the PIs' respective
        institutions. Many of the PIs on this project are located in the Chicago area,
        giving postdoctoral researchers a unique opportunity to network with each
        other, across institutions, and with PIs at other institutions (including
        through lectures and guest speaker engagements).
from local opportunities.

\item {\bf Travel.} This proposal includes travel funds
    to allow the postdoc to attend collaboration meetings conferences.  Active
participation through 
presentations will be encouraged.
\end{enumerate}
\noindent
Success of the mentoring plan will be assessed by monitoring the
personal progress of the postdoctoral researcher through a tracking of
the postdoctoral researcher's progress towards career goals
after finishing the postdoctoral program.
PIs on this project have successfully mentored postdocs who now have faculty
positions at computer science departments in the United States and abroad.
