\begin{center}
{\large \bf \TITLE}
\end{center}

\begin{center}
    {\bf Facilities, Equipment, and Other Resources}
\end{center}

The Department of Computer Science is the hub of a large, diverse computing
community of two hundred researchers focused on advancing foundations of
computing and driving its most advanced applications. Long distinguished in
theoretical computer science and artificial intelligence, the Department is
building a strong Systems research group. This closely knit community includes
the Data Science Institute (DSI), the Toyota Technological Institute, and
Argonne’s Mathematics and Computer Science Division.

In addition, as of August 2018, the Department of Computer Science moved into
its new home on the second and third floors of the renovated John Crerar
Library. This state-of-the-art academic space includes closed and open
offices, conference rooms, spaces for experimental research, graduate student
offices and a large gathering area for departmental seminars, workshops and
distinguished speaker lectures. The PI has access to a substantial collection
of computing resources at the University of Chicago including a large
visualization display and GVU clusters. The PIs receive administrative support
from the University of Chicago Physical Sciences Division, including contract
and grant accounting, meeting organization, and other administrative tasks.
The PIs also have access to a dedicated user studies laboratory at the
department for any in-laboratory, user studies, or design activities held at
the university as well as a graphic design studio, PSD Graphic Arts, housed in
the basement of the John Crerar Library for creating example content for user
studies and audit studies. All personnel involved in this project will be
given office and desk spaces on the campus of the University of Chicago.

Basic daily computing needs will be supplied partially by the Department of
Computer Science where the PIs are faculty members. Faculty members also have
access to the Research Computing Cluster (RCC) should the project team require
more computer power for any activities designed as part of the project. RCC
provides high-end research computing resources and support to researchers,
enabling the advancement of critical inquiry across a diverse set of
disciplines—the physical sciences, social sciences and humanities, medicine,
and economics. University researchers and their collaborators are provided
access to managed high-performance computing (HPC), storage, and visualization
resources. In addition to the cutting-edge hardware and the advanced software
that RCC maintains, its expert staff of computational scientists, research
programmers, application developers, and HPC systems administrators assist
researchers via consultation and training to help them fully leverage modern
HPC technology within their projects. The RCC’s HPC resources include the
Midway Compute Cluster housed in the 6045 Data Center. The Midway Cluster
forms the core of RCC’s advanced computational infrastructure.
